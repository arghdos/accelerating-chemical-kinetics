\documentclass{beamer}
\usepackage[style=numeric-comp, backend=bibtex]{biblatex}
\usepackage{lmodern}
\usepackage{amsmath}
\usepackage{mathtools}
\usepackage[utf8]{inputenc}

\graphicspath{{./Figures/}}
%smaller footnotes, adapted from http://tex.stackexchange.com/a/146021
\setbeamerfont{footnote}{size=\fontsize{7pt}{0pt}}

%auto incrementing titles
\newcounter{drgmethods}
\newcommand\DRGMethodsTitle{%
  \frametitle{\refstepcounter{drgmethods}DRG Methods~\thedrgmethods}}
\resetcounteronoverlays{drgmethods}


\bibliography{lit_review}
%\addbibresource{lit_review.bib}

%opening
\title{Accelerating Chemical Kinetic Evaluation: A Literature Review}
\author{Nick Curtis}
\institute{University of Connecticut}
\date{\today}

\begin{document}

\maketitle

\begin{frame}
\frametitle{Chemical Kinetic Simulations are \textbf{Critical}}
In order to meet increasingly stringent emissions and efficiency requirements, designers of combustion devices have turned to \textbf{new technologies} and \textbf{new fuels}
\begin{itemize}
 \item Novel combustion regimes such as low-temperature combustion are often controlled by chemical processes, rather than directly controllable physical processes as in current technology
 \item Further, developed solutions must be flexible to accommodate a variety of next generation fuels
\end{itemize}
Computationally guided combustion design has played an important role in development of these new technologies, however use of realistic chemical modeling (as required for predictive reacting-flow simulations) is prohibitively expensive for most practical systems. 
\end{frame}

\begin{frame}
 \frametitle{Chemical Kinetic Integration is \textbf{Expensive}}
 The size of chemical kinetic models for fuels relevant transportation and energy generation may consist of hundreds to thousands of chemical species, with potentially tens of thousands of reactions
 \begin{itemize}
  \item e.g., for gasoline \footfullcite{Mehl:2011cn} and jet fuel\footfullcite{Naik2011434}
 \end{itemize}
 Commonly used implicit integration integration techniques scales poorly with increasing model size.
 \begin{itemize}
  \item These methods require repeated evaluation and factorization of the chemical kinetic Jacobian.
  \item Naive implementations of these operations scale \textbf{quadratically} and \textbf{cubically} with the number of species in a model, respectively\footfullcite{Lu:2009gh}.
 \end{itemize}
\end{frame}

\begin{frame}
 \frametitle{Strategies for cost reduction}
 A few major strategies exist to accelerate chemical kinetic integration:
 \begin{itemize}
  \item \textbf{Model reduction}, a host of techniques to reduce the size of the system being solved while maintaining accurate chemical kinetics (as compared to the full model)
  \item \textbf{Improved integration techniques}, development of new integration algorithms specifically for chemical kinetics, e.g. hybrid implicit\slash explicit integrators, tabulation techniques, analytical Jacobian codes, and on-the-fly stiffness removal
  \item \textbf{Solver vectorization}, reformulation of solvers for efficient single-instruction, multiple-data (SIMD) execution both on specialized hardware---e.g. graphics processing units (GPUs)---and modern central processing units (CPUs)
  \item \textbf{High performance computing techniques}, stiffness-based load balancing, scaling for high performance clusters
 \end{itemize}
\end{frame}

\begin{frame}
 \frametitle{Model Reduction Techniques}
 Model reduction can be divided into a few broad categories:
 \begin{itemize}
  \item \textbf{Skeletal reduction} via elimination of unimportant species and reactions
  \item Simplification of the system and reduction of chemical stiffness via \textbf{time-scale analysis}
  \item \textbf{Lumping} of similar chemical species and reaction pathways to reduce model size
 \end{itemize}
\end{frame}

\begin{frame}
 \frametitle{Skeletal Model Reduction}
 Skeletal reduction removes species and reactions considered unimportant for a given thermochemical state-space from a detailed chemical kinetic model.
 Key features for skeletal model reduction techniques include\footfullcite{Lu:2009gh}:
 \begin{itemize}
  \item \textbf{\textit{A priori} error control}--to avoid expensive model validation after reduction
  \item \textbf{Efficient execution}--skeletal reduction techniques are often applied on-the-fly or as the first step in a reduction procedure, and must therefore be efficient for large models with many sampled thermochemical states
 \end{itemize}
\end{frame}
\begin{frame}
 \frametitle{Skeletal Reduction Techniques}
 Many different types of skeletal reduction techniques have been developed, including:
 \begin{itemize}
  \item Sensitivity analysis\footfullcite{rabitz:1983}, principal component analysis\footfullcite{vajda:1985}, level of importance analysis\footfullcite{lovas:2009} and computational singular perturbation (CSP) methods modified for skeletal reduction\footfullcite{valorani:2006}
  \item Directed relation graphs (DRG)\footfullcite{Lu:2006bb}, and related methods including DRG with error propagation (DRGEP)\footfullcite{Pepiot-Desjardins:2008} and DRG\slash DRGEP assisted sensitivity analysis\footfullcite{lu:2008}$^{,}$\footfullcite{niemeyer:2010}.
 \end{itemize}
\end{frame}
\begin{frame}
 \DRGMethodsTitle
 Of these, DRG and related techniques are often used as initial reduction methods in a comprehensive reduction process, or dynamic adaptive chemistry methods\footfullcite{liang:2009} due to their efficiency and reliability.
 \begin{itemize}
  \item Each species is considered a vertex on a graph
  \item Edges are determined by the strength of species interactions
 \end{itemize}
 \centering
  \begin{figure}
    \includegraphics[width=\linewidth,height=0.4\textheight,keepaspectratio]{DRGfig}
    \caption{A simple DRG example}
  \end{figure}
\end{frame}

\begin{frame}
 \DRGMethodsTitle
 In DRG, the species interactions coefficient is determined as:
 \begin{equation}
  r_{AB} \coloneqq \frac{\sum_{i=1\ldots N_R}\lvert \nu_{A,i} \dot{\omega}_i \delta_{B,i}\rvert}{\sum_{i=1\ldots N_R} \lvert\nu_{A,i} \dot{\omega}_i \rvert}
 \end{equation}
 where:
 \begin{itemize}
  \item $\nu_{A,i}$ is the net stoichiometric coefficient of species $A$ in reaction $i$
  \item $\dot{\omega}_i$ is the net reaction rate of reaction $i$
  \item $\delta_{B,i}$ is a Kronecker delta that is unity if species $B$ is present in reaction $i$
 \end{itemize}
 An edge is present on the DRG, if and only if $r_{AB} > \epsilon_{cut}$, where $\epsilon_{cut}$ is a user-specified cutoff threshold.
\end{frame}

\begin{frame}
 \DRGMethodsTitle
 Alternate formulations of the species interaction coefficient exist, e.g. to properly handle large isomer groups\footfullcite{luo:2010} or for DRGEP\footfullcite{Pepiot-Desjardins:2008}.
\end{frame}





\end{document}
